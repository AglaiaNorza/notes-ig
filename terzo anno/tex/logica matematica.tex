%! TEX program = xelatex
\documentclass[a4paper,12pt]{report}

\usepackage{./../packages/mainstyle}
\usepackage{./../packages/boxes}
\usepackage{./../packages/titleITA}

\setcoursename{Logica Matematica}
\setcoursebook{tbd}
\setauthorname{aglaia norza}
\setauthoremail{thisisaglaia@gmail.com}
\setauthorgithub{AglaiaNorza}

\begin{document}

\makefrontpage

\tableofcontents

\chapter{Logica Proposizionale}

\section{Introduzione}

Definiamo alcuni concetti fondamentali

\begin{defbox}{Linguaggio proposizionale}{}
    Un linguaggio proposizionale è un insieme di \textbf{variabili} e \textbf{proposizioni}.
\end{defbox}

\begin{defbox}{Proposizione}{}
    Una \textbf{proposizione} in un linguaggio proposizionale è un elemento dell'insieme PROP così definito:
    \begin{enumerate}
        \item tutte le variabili appartengono a PROP
        \item se \( A \in \) PROP, allora \( \neg A \in \) PROP
        \item se \( A, B \in \) PROP, allora \( (A \land B), (A \lor B), (A \implies B) \in \) PROP
        \item nient'altro appartiene a PROP {\color{gray}(PROP è il più piccolo insieme che contiene le variabili e soddisfa le proprietà di chiusura sui connettivi 1 e 2)}
    \end{enumerate}
\end{defbox}

\begin{defbox}{Assegnamento}{}
    Per un linguaggio \( \mathcal{L} \), un \textbf{assegnamento} è una funzione 
    \[
        \alpha : \mathcal{L} \to \{0, 1\}
    \]

    Estendiamo \( \alpha \) ad \( \hat{\alpha} : \text{PROP} \to \{0,1\} \) in questo modo:
    \begin{itemize}
        \item 
    \end{itemize}

    
\end{defbox}

\end{document}
