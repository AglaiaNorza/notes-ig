%! TEX program = xelatex
\documentclass[a4paper,11pt]{report}

\usepackage{./../packages/mainstyle}
\usepackage{./../packages/boxes}
\usepackage{./../packages/titleITA}

\usepackage{float}

\setcoursename{Logica Matematica}
\setcoursebook{tbd}
\setauthorname{aglaia norza}
\setauthoremail{thisisaglaia@gmail.com}
\setauthorgithub{AglaiaNorza}

\settitlecolor{teal!30}

\begin{document}

\makefrontpage

\tableofcontents

\chapter{Logica Proposizionale}

\section{Introduzione}

La logica proposizionale è un linguaggio formale con una semplice struttura sintattica basata su proposizioni elementari (atomiche) e sui seguenti connettivi logici:


\begin{itemize}
    \item \textit{Negazione} ($\neg$): inverte il valore di verità di un enunciato: se un enunciato è vero, la sua negazione è falsa, e viceversa.

    \item \textit{Congiunzione} ($\land$): il risultato è vero se e solo se entrambi i componenti sono veri.

    \item \textit{Disgiunzione} ($\lor$): il risultato è vero se almeno uno dei componenti è vero.

    \item \textit{Implicazione} ($\to$): rappresenta l’enunciato logico “se ... allora”. Il risultato è falso solo se il primo componente è vero e il secondo è falso. 

    \item \textit{Equivalenza} ($\leftrightarrow$): rappresenta l’enunciato logico “se e solo se”. 
        Il risultato è vero quando entrambi i componenti hanno lo stesso valore di verità, cioè sono entrambi veri o entrambi falsi.
\end{itemize}

Introduciamo anche il concetto di disgiunzione esclusiva o "XOR" (\( \oplus \)), il cui risultato è vero solo se gli operandi sono diversi tra di loro (uno vero e uno falso).

\begin{defbox}{Linguaggio proposizionale}{}
    Un linguaggio proposizionale è un insieme infinito \( \mathcal{L} \) di simboli detti \textbf{variabili proposizionali}, tipicamente denotato come \( \{p_i : i \in I\} \) {\color{gray} (con \( I \) "insieme di indici")}.
\end{defbox}

\begin{defbox}{Proposizione}{}
    Una \textbf{proposizione} in un linguaggio proposizionale è un elemento dell'insieme PROP così definito:
    \begin{enumerate}
        \item tutte le variabili appartengono a PROP
        \item se \( A \in \) PROP, allora \( \neg A \in \) PROP
        \item se \( A, B \in \) PROP, allora \( (A \land B), (A \lor B), (A \to B) \in \) PROP
        \item nient'altro appartiene a PROP {\color{gray}(PROP è il più piccolo insieme che contiene le variabili e soddisfa le proprietà di chiusura sui connettivi 1 e 2)}
    \end{enumerate}
\end{defbox}

Per facilitare la leggibilità delle formule, definiamo le seguenti regole di \textit{precedenza}: \( \neg \) ha precedenza su \( \land, \lor \), e questi ultimi hanno precedenza su \( \to \).

\section{Assegnamenti, tavole di verità}

Per un linguaggio \( \mathcal{L} \), un \textbf{assegnamento} è una funzione 
\[
    \alpha : \mathcal{L} \to \{0, 1\}
\]

Estendiamo \( \alpha \) ad \( \hat{\alpha} : \text{PROP} \to \{0,1\} \) in questo modo:
\vspace{0.5em}
\begin{itemize}
    \item \( \hat{\alpha}(\neg A) = \begin{cases}
            1 &  A = 0 \\
            0 & A = 1
        \end{cases}\)

        \vspace{0.5em}

    \item \( \hat{\alpha}(A \land B) = \begin{cases}
            1 & \hat{\alpha}(A) = \hat{\alpha}(B)  = 1 \\
            0 & altrimenti
        \end{cases}\)

        \vspace{0.5em}

    \item \( \hat{\alpha}(A \lor B) = \begin{cases}
            0 & \hat{\alpha}(A) = \hat{\alpha}(B)  = 0 \\
            1 & altrimenti
        \end{cases}\)

        \vspace{0.5em}

    \item \( \hat{\alpha}(A \to B) = \begin{cases}
            0 & \hat{\alpha}(A) = 1 \land \hat{\alpha}(B)  = 0 \\
            1 & altrimenti
        \end{cases}\)

\end{itemize}

\begin{gbox}{notazione}
    Utilizzeremo \( \alpha \) al posto di \( \hat{\alpha} \) per comodità di notazione.
\end{gbox}

Osserviamo che è possibile rappresentare gli assegnamenti in modo compatto utilizzando le \textbf{tavole di verità}, una presentazione tabulare della funzione di assegnamento.

Per esempio, possiamo riscrivere la definizione di \( \alpha(\neg A) \) come segue:

\[  
    \begin{array}{c | c}
        A & \neg A \\
        \hline
        0 & 1 \\
        1 & 0

    \end{array}
\]

Ogni riga di una tavola di verità corrisponde ad un assegnamento \( \alpha \).

Si noti anche che dalla definizione di \( \alpha \) segue che un'implicazione può essere vera senza che ci sia connessione causale o di significato tra antecedente e conseguente (per esempio, "se tutti i quadrati sono pari allora \( \pi \) è irrazionale"). 

In secondo luogo, segue anche che una proposizione è sempre vera se il suo antecedente è falso (il che rispecchia la pratica matematica di considerare vera a vuoto una proposizione ipotetica la cui premessa non si applica).


{\color{CadetBlue} Questo è giustificabile come segue:
    \begin{itemize}
        \item vogliamo che \( (A \land B) \to B \) sia sempre vera
        \item il caso \( 1 \to 1 \) deve essere vero, perché corrisponde al caso in cui \( A \) e \( B \) sono vere; 

            il caso \( 0 \to 0  \) deve essere vero, perché corrisponde al caso in cui \( A\land B \) è falso perché \( B \) è falso; il caso \( 0 \to 0 \) deve essere vero perché corrisponde al caso in cui \( A \land B \) è falso perché \( B \) è falso; 

            il caso \( 0 \to 1 \) deve essere vero perché corrisponde al caso in cui \( A \land B \) è falso perché \( A \) è falso ma \( B \) è vero; 

            resta dunque soltanto il caso \( 1 \to 0 \), che non corrisponde a nessun caso di \( A \land B \to B \).
    \end{itemize}

In più, si vuole che valga, per contrapposizione \( (A \to B)\to(\neg B \to \neg A) \).}


Osserviamo che, data \( A = p_1, p_2, \dots, p_k \) e due assegnamenti \( \alpha \) e \( \beta \) t.c.:
\begin{align*}
    \alpha(p_1) &= \beta(p_1) \\
                &\dots \\
    \alpha(p_k) &= \beta(p_k)
\end{align*}

allora necessariamente \( \alpha(A) = \alpha(B) \).

\begin{gbox}[colframe=PineGreen]{soddisfacibilità}
    Se per una formula \( A \) e un assegnamento \( \alpha \) si ha \( \alpha (A) = 1 \), si dice che ``\(A \) soddisfa \( \alpha \)'' (o ``\( A \) è vera sotto \( \alpha \)'').
    \begin{itemize}
        \item Se \( A \) ha almeno un assegnamento che la soddisfa, si dice \textbf{soddisfacibile} (\( A \in \texttt{SAT} \)).
        \item Se non esiste un assegnamento che la soddisfa, \( A \) si dice \textbf{insoddisfacibile} (\( A \in \texttt{UNSAT} \)).
        \item Se \( A \) è soddisfatta da tutti i possibili assegnamenti, si dice \textbf{tautologia} (o "verità logica") (\( A \in \texttt{TAUT} \)).
    \end{itemize}
\end{gbox}

Introduciamo anche alcune regole che 

\section{Conseguenza logica}

\begin{defbox}{Conseguenza logica}{}
    Sia \( T \) una \textit{teoria}, ossia un insieme  \( \{A_1, \dots, A_n\} \) proposizioni in un dato linguaggio proposizionale, e sia \( A \in \text{PROP}\) .

    Diciamo che \( A \) è \textbf{conseguenza logica} di \( T\) se 
    \[ \forall \alpha,\ \alpha(T)=1 \to \alpha(A)=1 \] 
    ovvero se ogni assegnamento che soddisfa \(T\) soddisfa anche \( A_{n+1} \).

    Scriviamo in tal caso \(  T \vDash A_{n+1} \), oppure \( A_1, \dots, A_n \vDash A \).
\end{defbox}

Si ha che:
\begin{itemize}
    \item \(T \not\vDash A\) \ significa che \ \( \exists \alpha \) \ t.c. \ \( \alpha(T) = 1 \land \alpha(A) = 0 \)
    \item \( \emptyset \vDash A \) \ o, equivalentemente \ \( \vDash A \iff A\) è una tautologia
\end{itemize}

\begin{lemmabox}{Equivalenze}{}
    \begin{enumerate}
        \item \( T \vDash A \)
        \item \( \vDash (A_1 \land \dots \land A_n) \to A \)
        \item \( (A_1 \land \dots \land A_n) \in \texttt{UNSAT}\) 
    \end{enumerate}

    sono equivalenti.

\end{lemmabox}

\section{Completezza funzionale}
\textit{Data una tavola di verità arbitraria con \( n \) argomenti, esiste una proposizione \( A \) che ha esattamente quella tavola di verità?}

Una proposizione \( A \) contenente le \( n \) variabili proposizionali \( a_1, a_2, \dots, a_n \) determina una funzione di \( n \) argomenti \( f: \{0, 1\}^n \to \{0,1\} \) ("\textbf{funzione di verità}"), tale che il valore di \( f_A \) su un argomento \ \( (x_1, x_2, \dots, x_n) \in \{0,1\}^n\) \ sia dato da un arbitrario assegnamento \( \alpha \) tale che \( \alpha(p_k) = x_k\) \ per \ \(k \in [1,n] \).

\begin{thmbox}{Teorema}{}
    Sia \( f: \{0, 1\}^n \to \{0,1\} \) una funzione di verità. Esiste una proposizione \( A \) con \( n \) variabili proposizionali tale che, per ogni assegnamento \( \alpha \):
    \[ \alpha(A) = f(\alpha(a_1), \alpha(a_2), \dots, \alpha(a_n)) \]
\end{thmbox}

\begin{proofbox}[title=dimostrazione]           
    Si dimostra per induzione su \( n \).

    \begin{itemize}
        \item \textbf{caso base}: \( n=1 \)
            abbiamo quattro possibili \( f \): 
            \[
                \begin{aligned}
                    f_1(0) &= 0, \quad f_1(1) = 0 \\
                    f_2(0) &= 1, \quad f_2(1) = 1 \\
                    f_3(0) &= 0, \quad f_3(1) = 1 \\
                    f_4(0) &= 1, \quad f_4(1) = 0
                \end{aligned}
            \]

            Alla funzione $f_1$ corrisponde la formula $(p \land \neg p)$, alla funzione $f_2$ la formula $(p \lor \neg p)$, 
            alla funzione $f_3$ la formula $p$, e alla funzione $f_4$ la formula $(\neg p)$.
        \item \textbf{caso induttivo}: (assumiamo che il teorema valga per \( n-1 \) variabili, e dimostriamo che vale per \( n \))

            Se $n > 1$, scriviamo il grafico di 
            \[
                f : \{0,1\}^n \to \{0,1\}
            \]

            in forma di tavola di verità in questo modo:

            \[
                \begin{array}{cccc|c|l}
                    p_1 & p_2 & \cdots & p_n & f(p_1, \ldots, p_n) & \\ \hline
                    0 & \cdots & \cdots & 0 & \cdots &  \\
                    \vdots & & & \vdots & \vdots & \text{grafico di una funzione } f_0\\
                    0 & \cdots & \cdots & 1 & \cdots & \\ \hline
                    1 & \cdots & \cdots & 0 & \cdots &  \\
                    \vdots & & & \vdots & \vdots & \text{grafico di una funzione } f_1\\
                    1 & \cdots & \cdots & 1 & \cdots & 
                \end{array}
            \]

            Se non consideriamo la prima colonna (\( p_1 \)), la tavola di verità descrive il grafico di due funzioni, \( f_0 \) e \( f_1 \), a \( n-1 \) argomenti.

            Sappiamo, quindi, per ipotesi induttiva, che esistono due formule \( A_0 \) e \( A_1 \) a \( n-1 \) variabili tali che, per ogni assegnamento \( \alpha \):
            \[ \alpha(A_0) = f_0(\alpha(p_1), \alpha(p_2), \dots, \alpha(p_n)) \]
            \[ \alpha(A_1) = f_1(\alpha(p_1), \alpha(p_2), \dots, \alpha(p_n)) \]

            Dobbiamo ora combinare le due formule considerando anche la colonna \( p_1 \).

            Possiamo farlo tramite la formula \(A= (\neg p_1 \to A_0) \land (p_1 \to A_1) \).

            Dimostriamo che \( A \) soddisfa il teorem: dobbiamo dimostrare che, dato un assegnamento qualsiasi \( \alpha \), si ha:
    \[ \alpha(A) = f(\alpha(p_1), \alpha(p_2), \dots, \alpha(p_n)) \]

    Distinguiamo i due casi:
    \begin{itemize}
        \item \( \alpha(p_1) = 1 \) 

            in questo caso, si ha:
            \[
                \alpha\!\left(
                    \underset{=1}{(\neg p_1 \to A_0)}
                    \land
                \underset{=1}{(p_1} \to A_1)
                \right)
            \]

            e la formula vale quindi \( 1 \iff \alpha(A_1) = 1 \).

            Ma \( \alpha(A_1) = f_1(\alpha(p_2), \dots, \alpha(p_n)) \), quindi la formula si comporta esattamente come \( f_1 \):
            \[
                f(\alpha(p_1), \alpha(p_2), \ldots, \alpha(p_n))
                = f(1, \alpha(p_2), \ldots, \alpha(p_n))
                = f_1(\alpha(p_2), \ldots, \alpha(p_n)).
            \]

            Quindi, in questo caso, vale 
            \[\alpha(A) = (\alpha(p_1), \alpha(p_2), \ldots, \alpha(p_n))\]

        \item \( \alpha(p_1) = 0 \)            

            in questo caso, si ha:
            \[
                \alpha\!\left(
                    \underset{=1}{(\neg p_1} \to A_0)
                    \land
                    \underset{=1}{(p_1 \to A_1})
            \right)\]

            che vale \( 1 \iff \alpha(A_0)=1\).

            Quindi si può fare lo stesso ragionamento di sopra, ma per \( A_1 \) e \( f_0 \).

            \begin{gbox}{}
                Potremmo anche costruire una funzione \( f \) che rappresenta il comportamento di \( A \):

                \[ f(x_1, x_2, \ldots, x_n) =
                    \begin{cases}
                        f_1(x_2, \ldots, x_n) & \text{se } x_1 = 1, \\
                        f_0(x_2, \ldots, x_n) & \text{se } x_1 = 0.
                    \end{cases}
                \]
            \end{gbox}

    \end{itemize}

\end{itemize}

\end{proofbox}

\section{Forme normali}

\begin{gbox}{notazione}
    Chiamiamo "letterale" una variabile proposizionale o una negazione di una variabile proposizionale
\end{gbox}

È utile individuare alcune forme normali canoniche.

\begin{defbox}{Forma Normale Disgiuntiva}{}
    Diciamo che \( A \) è in Forma Normale Disgiuntiva (\textbf{DNF}, \textit{Disjunctive Normal Form}) se \( A \) è una disgiunzione di congiunzioni di letterali, ossia è nella forma seguente:

    \[ \bigvee_{i\leq n} \bigwedge_{j\leq m_i} A_{ij} = (A_{1,1} \land \dots \land A_{1, m_1}) \lor \dots \lor (A_{n,1} \land \dots \land A_{n, m_n}) \]

\end{defbox}

\begin{defbox}{Forma Normale Congiuntiva}{}
    Diciamo che \( A \) è in Forma Normale Congiuntiva (\textbf{CNF}, \textit{Conjunctive Normal Form}) se \( A \) è una disgiunzione di congiunzioni di letterali, ossia è nella forma seguente:

    \[ \bigwedge_{i\leq n} \bigvee_{j\leq m_i} A_{ij} = (A_{1,1} \lor \dots \lor A_{1, m_1}) \land \dots \land (A_{n,1} \lor \dots \lor A_{n, m_n}) \]

\end{defbox}

\section{Equivalenza Logica}

\begin{defbox}[colframe=PineGreen, colback=DeepGreenLight]{Equivalenza logica}{}
    Due formule \( A, B \in \text{PROP} \) sono logicamente equivalenti (\( A \equiv B \)) quando, per ogni assegnamento \( \alpha \) si ha \( \alpha(A) = \alpha(B) \).
\end{defbox}


Introduciamo alcune regole utili per verificare l'equivalenza tra proposizioni.

Con un piccolo abuso di notazione, definiamo \( 1 \) e \( 0 \) come le formule per cui \( \forall \alpha, \ \alpha(1)= 1 \) e \( \alpha(0) = 0 \).

In questo modo, abbiamo:

\begin{table}[H]
    \centering
    \renewcommand{\arraystretch}{1.3}
    \begin{tabular}{|l|l|}
        \hline
        \textbf{Involuzione} & $\neg\neg A \equiv A$ \\
        \hline
        \textbf{Assorbimento (con 0 e 1)} &
        $A \lor 0 \equiv A$ \\
                                          & $A \land 1 \equiv A$ \\
                                          \hline
        \textbf{Cancellazione} &
        $A \lor 1 \equiv 1$ \\
                               & $A \land 0 \equiv 0$ \\
                               \hline
        \textbf{Terzo escluso (\textit{tertium non datur})} &
        $A \lor \neg A \equiv 1$ \\
                                                            & $A \land \neg A \equiv 0$ \\
                                                            \hline
        \textbf{Leggi di De Morgan} &
        $\neg(A \lor B) \equiv \neg A \land \neg B$ \\
                                    & $\neg(A \land B) \equiv \neg A \lor \neg B$ \\
                                    \hline
        \textbf{Commutatività} &
        $A \lor B \equiv B \lor A$ \\
                               & $A \land B \equiv B \land A$ \\
                               \hline
        \textbf{Associatività} &
        $A \lor (B \lor C) \equiv (A \lor B) \lor C$ \\
                               & $A \land (B \land C) \equiv (A \land B) \land C$ \\
                               \hline
        \textbf{Distributività} &
        $A \lor (B \land C) \equiv (A \lor B) \land (A \lor C)$ \\
                                & $A \land (B \lor C) \equiv (A \land B) \lor (A \land C)$ \\
                                \hline
        \textbf{I teorema di assorbimento} &
        $A \lor (A \land B) \equiv A$ \\
                                           & $A \land (A \lor B) \equiv A$ \\
                                           \hline
        \textbf{II teorema di assorbimento} &
        $A \lor (\neg A \land B) \equiv A \lor B$ \\
                                            & $A \land (\neg A \lor B) \equiv A \land B$ \\
                                            \hline

    \end{tabular}
    \caption{Principali leggi di equivalenza logica}
\end{table}

\section{Formalizzazioni in logica proposizionale}

Il concetto di soddisfacibilità ci permette di usare insiemi di formule proposizionali per catturare determinate strutture matematiche.

Per esempio: sia \( X \) un insieme. Consideriamo il linguaggio proposizionale composto dalle variabili \(p_{(x, y)}  \) per ogni \( (x,y) \in X \times X\), e consideriamo il seguente insieme \( T \) di proposizioni in questo linguaggio:

\begin{enumerate}
    \item \( \neg p_{x,x} \ \ \forall x \in X\) \ {\color{gray}(antiriflessività)}
    \item \( p_{x, y} \to \neg p_{y,x} \ \   \forall x \in X\) \ {\color{gray}(asimmetria)}
    \item \( (p_{x, y} \land p_{y, z}) \to p_{x, z} \ \ \forall x, y, z \in X\) \ {\color{gray}(transitività)}
\item \( (p_{x, y} \lor p_{y, x}) \  \ \forall x \neq  y \in X\) \ {\color{gray}(ordine totale)}
\end{enumerate}

Usiamo una teoria \( T \) per poter gestire anche casi di insiemi infiniti. Infatti, sappiamo che una teoria infinita è soddisfatta se e solo se lo sono tutte le sue proposizioni.

L'insieme \( T = T_X \) esprime il concetto di \textbf{ordine totale stretto} su \( X \). Infatti, se avessimo un assegnamento \( \alpha \) che soddisfa tutte le proposizioni di \( T \), l'ordine indotto da tutte le variabili vere sotto \( \alpha \) sarebbe un ordine totale stretto di \( X  \).

Se \( \alpha \) è un assegnamento, definiamo la relazione \( \prec_{\alpha} \) su \( X \) come segue:
\[ x \prec_\alpha y \leftrightarrow \alpha(p_{x,y})=1 \]

Si ha che per ogni assegnamento \( \alpha \) che soddisfca \( T_X \), l'ordine \( \prec_\alpha \) indotto da \( \alpha \) è un ordine totale stretto su \( X \).

Dall'altra parte, se \( \prec \) è un ordine totale stretto su \( X \), e \( \alpha_\prec \) è l'assegnamento indotto da \( \prec \) così definito:
\[ \alpha_\prec (p_{x,y}) = 1 \leftrightarrow (x \prec y) \]

Si ha che, per ogni ordine totale stretto \( \prec \) su \(X \), l'assegnamento \( \alpha_\prec \) indotto da \( \prec \) sulle variabili \( p_{x, y} \) soddisfa \( T \).

Ovvero, un assegnamento \( \alpha \) soddisfa la teoria \( T_X \) se e solo se l'ordine indotto da \( \alpha \) su \( X \) è un ordine totale.

\begin{gbox}{Colorabilità}

\end{gbox}

\section{Teorema di compattezza}

\begin{defbox}{Monotonia della conseguenza logica}{}
    Si dice che la nozione di conseguenza logica è \textbf{monotona}, ovvero che \[T' \vDash A \land T' \subseteq T \implies T \vDash A\]
    (se \( A_1, A_2, \dots, A_k \vDash A \), allora \( T \vDash A \) per ogni teoria \( T \) contenente \( A_1, A_2, \dots, A_k \))
\end{defbox}

Nonostante non sembri intuitivamente vero, vale anche il viceversa:

\begin{thmbox}{Teorema di compattezza v.1}{}
    Se \( T \vDash A \), esiste un sottoinsieme finito \( T_0 \) di \( T \) tale che \( T_0 \vDash A \)
\end{thmbox}

Introduciamo il concetto di una teoria finitamente soddisfacibile:

\begin{defbox}{\texttt{FINSAT}}{}
    Una teoria si dice \textbf{finitamente soddisfacibile} (\( \in \texttt{FINSAT} \)) se \textit{ogni} suo sottoinsieme finito è soddisfacibile.
\end{defbox}

Possiamo quindi introdurre una nuova versione del teorema di compattezza:

\begin{thmbox}{Teorema di compattezza v.2}{}
    \( \texttt{FINSAT}\implies \texttt{SAT} \), ovvero se ogni sottoinsieme di \( T \) è soddisfacibile, anche \( T \) è soddisfacibile.
\end{thmbox}

\begin{lemmabox}{Teorema di compattezza v.1 \( \equiv \) v.2}{}
    I due punti seguenti (le due versioni del teorema di compattezza) sono equivalenti:
    \begin{enumerate}
        \item \( T \vDash A \iff \exists \ T_0 \overset{fin}{\subseteq} T \ t.c. \  T_0 \vDash A\)
        \item \( T \in \texttt{SAT} \iff T \in \texttt{FINSAT} \)
    \end{enumerate}

    \begin{proofbox}[title=dim.]{}
        \begin{itemize}
            \item \textcircled{1} \( \implies \) \textcircled{2}

                Supponiamo per assurdo che \( T \in \texttt{FINSAT} \implies T \in \texttt{SAT} \), e che \( T \vDash A \) ma che \( \forall T_0 \overset{fin}{\subseteq } T, \ T_0 \not\vDash A \).

                \( T \not\vDash A \) significa \( T \cup \{\neg A\} \in \texttt{SAT} \).
                
                Quindi, visto che \( \texttt{FINSAT} \implies \texttt{SAT} \), \( T \cup \{ \neg A\} \in \texttt{SAT}\), il che va in contraddizione con l'ipotesi \( T \vDash A \).
            \item \textcircled{2} \( \implies \) \textcircled{1}

                Supponiamo per assurdo che \( T\vDash A \implies \exists T_0  \overset{fin}{\subseteq } T \ t.c. \ T_0 \vDash A\), che \( T\in \texttt{FINSAT} \), ma che \( T\not \in \texttt{SAT} \) (\( T\in \texttt{UNSAT} \)).

                Se \( T \in \texttt{UNSAT} \), possiamo dire che \( T \vDash p \land \neg p \) {\small \color{gray} (tutto è conseguenza logica di una teoria insoddisfacibile)}.

                Per \textcircled{2}, quindi, \( \exists T_0 \ t.c, \ T_0 \overset{fin}{\subseteq } T \vDash p \land \neg p \), il che va in contraddizione con \( T \in \texttt{FINSAT} \).\qed
        \end{itemize}

    \end{proofbox}

\end{lemmabox}

\begin{thmbox}{Estendibilità di \texttt{SAT}}{}
    Se \( T \) è soddisfacibile, allora \( T \cup \{A\} \) è soddisfacibile oppure \( T \cup \{\neg A\} \) è soddisfacibile.

    \begin{proofbox}[title=dimostrazione dalle dispense]{}
        Sia \( \alpha \) un assegnamento che soddisfa \( T \). Se \( \alpha(A)=1 \) allora \( T \cup \{A\} \) è soddisfacibile. Se \( \alpha(A) = 0 \),  \( T \cup \{\neg A\} \) è soddisfacibile.
    \end{proofbox}
    \begin{proofbox}[title=dimostrazione vista in classe]{}
        Supponiamo \( T \in \texttt{SAT} \), \ \( T \cup \{A\} \in \texttt{UNSAT}\) \ e \ \( T \cup \{\neg A\} \in \texttt{UNSAT}\). Avremmo entrambi \( T \vDash \{\neg A\}\) e \( T \vDash A \), il che è impossibile se \( T\in \texttt{SAT} \).
    \end{proofbox}
\end{thmbox}

Un concetto analogo vale per \texttt{FINSAT}.

\begin{thmbox}{Estendibilità di \texttt{FINSAT}}{}
    Sia \( T \in \texttt{FINSAT} \). Per ogni formula \( A, \  T \cup \{A\} \in \texttt{FINSAT}\)  \ o \( \ T \cup \{\neg A\} \in \texttt{FINSAT}\)
    \begin{proofbox}[title=dim]{}
        Supponiamo per assurdo che \( T \cup \{A\} \not\in \texttt{FINSAT}\) e \( T \cup \{\neg A\} \not\in \texttt{FINSAT}\).

        Vuol dire che esistono \( B \ \overset{fin}{\subseteq} T \cup \{A\}\) \ e \ \( C \ \overset{fin}{\subseteq} T \cup \{\neg A\}\) insoddisfacibili.

        Dato che per ipotesi \( T \in \texttt{FINSAT} \), sappiamo che \( A \in B, C \). Possiamo quindi introdurre \( \hat B = B \setminus \{A\}\) e \( \hat C = C \setminus \{A\}\).

        Sappiamo che l'insieme \( \hat B \cup \hat C  \in \texttt{FINSAT}\), in quanto sottoinsieme finito di \( T \).

        Sia \( \alpha \) un assegnamento che lo soddisfa. Se \( \alpha(A)=1 \), allora soddisfa anche \( B \). Se \( \alpha(A) = 0 \), soddisfa anche \( C \). In entrambi i casi abbiamo una contraddizione.

    \end{proofbox}
\end{thmbox}





\end{document}

